% !TEX encoding = utf-8
\documentclass[12pt, a4paper]{article}
\usepackage[polish]{babel} 
\usepackage[utf8x]{inputenc}
\usepackage{graphicx}
\usepackage{booktabs}
\usepackage{amsmath, amsfonts}
\usepackage[T1]{fontenc}
\usepackage{polski}
\usepackage{multicol}
\usepackage{tgtermes}
\usepackage{xcolor}
\usepackage{tabularx}
\usepackage[margin=2cm]{geometry}
\usepackage{multirow}
\usepackage{listings}
\usepackage{verbatim}
\usepackage{verbatimbox}
\usepackage{hyperref}

\begin{document}

\begin{table}[!htb]
\centering
\begin{tabular}{|p{8cm}|p{7cm}|} \hline
			\multicolumn{2}{|l|}{ } \\
			\multicolumn{2}{|l|}{ Politechnika Warszawska}\\
			\multicolumn{2}{|l|}{Wydział Geodezji i Kartografii}\\  
			\multicolumn{2}{|l|}{Zakład Geodezji i Astronomii Geodezyjnej}\\  
			\multicolumn{2}{|l|}{ } \\
			\multicolumn{2}{|l|}{\textbf{Przedmiot: Informatyka Geodezyjna II }}\\ 
			\multicolumn{2}{|l|}{ } \\ \hline 
			\multicolumn{2}{|l|}{ } \\
			\multicolumn{2}{|l|}{ } \\
			\multicolumn{2}{|l|}{ } \\
			\multicolumn{2}{|l|}{ } \\
			\multicolumn{2}{|l|}{ } \\
			\multicolumn{2}{|l|}{ } \\
			\multicolumn{2}{|l|}{ } \\
			\multicolumn{2}{|l|}{ } \\
			\multicolumn{2}{|l|}{ } \\
			\multicolumn{2}{|l|}{ } \\
			\multicolumn{2}{|l|}{ } \\
			\multicolumn{2}{|l|}{ } \\
			\multicolumn{2}{|l|}{\LARGE{\textbf{Ćwiczenie nr 1}}}\\ 
			\multicolumn{2}{|l|}{ } \\
			\multicolumn{2}{|l|}{\Large{Stworzenie programu transformującego współrzędne pomiędzy układami}}\\  
			\multicolumn{2}{|l|}{\Large}\\
			\multicolumn{2}{|l|}{ } \\
			\multicolumn{2}{|l|}{ } \\
			\multicolumn{2}{|l|}{\small{Oświadczam, że niniejsza praca stanowiąca podstawę do uznania osiągnięcia efektów uczenia się z}} \\
			\multicolumn{2}{|l|}{\small{przedmiotu \textbf{Informatyka Geodezyjna II} została wykonana przeze mnie samodzielnie}} \\
			\multicolumn{2}{|l|}{ } \\
			\multicolumn{2}{|l|}{ } \\
			\multicolumn{2}{|l|}{ } \\
			\multicolumn{2}{|l|}{ } \\
			\multicolumn{2}{|l|}{ } \\
			\multicolumn{2}{|l|}{ } \\
			\multicolumn{2}{|l|}{ } \\
			\multicolumn{2}{|l|}{ } \\
			\multicolumn{2}{|l|}{ } \\
			\multicolumn{2}{|l|}{ } \\
			\multicolumn{2}{|l|}{ } \\ \hline
			\ & \ \\
			\ & \ \\
			\textbf{Sebastian Hetel, Julia Białas} & Data wydania ćwiczenia: 03-04-2023 \\
			\ & \ \\
			\textbf{Numer indeksu: 319320, 319293}& Data oddania ćwiczenia: 28-04-2023 \\
			\ & \ \\
			Grupa: 1 & Zajęcia: poniedziałek 12:15-14:00 \\
			Semestr: 4 & \\
			Rok akademicki: 2022/2023 &  Prowadzący: mgr inż. Andrzej Szeszko \\
			\ & \ \\ \hline
			
			
\end{tabular}
		
\end{table}
	
\newpage
\section{Cel ćwiczenia}
Celem ćwiczenia było utworzenie programu, w języku Python, który będzie przeliczał współrzędne pomiędzy różnymi układami i elipsoiami. Celem było również umożliwienie odczytu wielu współrzędnych z podanego pliku i zapisane do nowego. Ważne również było zaimplementowanie możliwoście uruchomienia kodu za pomocą wiersza poleceń przy pomocy biblioteki Argparse.  
\subsection{Wykorzystane narzędzia}
Do wykonania programu zostały urządzenia z różnymi systemami operacyjnymi
\begin{itemize}
\item Windows 10
\item MacOS
\end{itemize} 
Programami, które były wykorzystane do utworzenia skryptu oraz testwania go były:
\begin{itemize}
\item Visual Studio Code
\item NotePad ++
\item Notatnik
\end{itemize} 
\section{Przebieg ćwiczenia}
Na początku stworzyliśmy klase Transformations, w której zaczęliśmy tworzyć metody które pozwalają na transforamcje współrzędnych pomiędzy pomiędzy układami.\\
Pierwszą stworzoną metodą była metoda hirvonen, którą napisaliśmy na podstawie 
\url{http://www.asgeupos.pl/index.php?wpg_type=tech_transf&sub=xyz_blh}\\
Następnie stworzyliśmy funkcję odwrotną do hirvonena czyli flh2xyz, również na podstawie na podstawie 
\url{http://www.asgeupos.pl/index.php?wpg_type=tech_transf&sub=xyz_blh}\\
Kolejnym krokiem było zaimplementowanie metody neu, pozwalającej na przeliczenie współrzędnych geocentrycznych na współrzędne topocentryczne.



\end{document}